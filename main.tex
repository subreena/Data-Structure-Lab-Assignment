\documentclass{article}
\usepackage{array}
\usepackage{amsmath}
\usepackage[pdftex]{pict2e}
\usepackage[utf8]{inputenc}

\title{\textbf{Data Structure: Theoretical Approach}}
\author{\textbf{Durgesh Raghuvanshi}\\B-Tech Department of Computer Science,\\
IILM Academy of Higher Learning, Greater Noida, Uttar Pradesh, India}

\date{19 May 2022}
\begin{document}

\maketitle

\section*{ABSTRACT}
Run with accordance with significance. The first if 
these this paper explains about the basic terminologies 
used in this paper in data structure. Better running 
times will be other constraints, such as memory use 
which will be paramount. The most appropriate data 
structures and algorithms rather than through hacking 
removing a few statements by some clever coding. 
Data structures serve as the basis for abstract data 
types (ADT). "The ADT defines the logical form of 
the data type. The data structure implements the 
physical form of the data type."Different types of data 
structures are suited to different kinds of applications, 
and some are highly specialized to specific tasks. For 
example, relational databases commonly use B-tree 
indexes for data retrieval, while compiler 
implementations usually use hash tables to look up 
identifiers. 

\section*{INTRODUCTION}
Data structures serve as the basis for abstract data 
types (ADT). "The ADT defines the logical form of 
the data type. The data structure implements the 
physical form of the data type."Different types of data 
structures are suited to different kinds of applications, 
and some are highly specialized to specific tasks. For 
example, relational databases commonly use B-tree 
indexes for data retrieval, while compiler 
implementations usually use hash tables to look up 
identifiers. Data structures provide a means to manage 
large amounts of data efficiently for uses such as large 
databases and internet indexing services. Usually, 
efficient data structures are key to designing efficient 
algorithms. Some formal design methods and 
programming languages emphasize data structures, 
rather than algorithms, as the key organizing factor in 
software design. Data structures can be used to 
organize the storage and retrieval of information 
stored in both main memory and secondary memory. 
Data structures are generally based on the ability of a 
computer to fetch and store data at any place in its 
memory, specified by a pointer—a bit string, 
representing a memory address, that can be itself 
stored in memory and manipulated by the program. 
Thus, the array and record data structures are based on 
computing the addresses of data items with arithmetic 
operations, while the linked data structures are based 
on storing addresses of data items within the structure 
itself. Many data structures use both principles, 
sometimes combined in non-trivial ways (as in XOR 
linking).[citation needed] 
The implementation of a data structure usually 
requires writing a set of procedures that create and 
manipulate instances of that structure. The efficiency 
of a data structure cannot be analyzed separately from 
those operations. This observation motivates the 
theoretical concept of an abstract data type, a data 
structure that is defined indirectly by the operations 
that may be performed on it, and the mathematical 
properties of those operations (including their space 
and time cost).[citation needed]An array is a number 
of elements in a specific order, typically all of the 
same type (depending on the language, individual 
elements may either all be forced to be the same type, 
or may be of almost any type). Elements are accessed 
using an integer index to specify which element is 
required. Typical implementations allocate contiguous 
memory words for the elements of arrays (but this is 
not necessity). Arrays may be fixed-length or 
resizable. A linked list (also just called list) is a linear 
collection of data elements of any type, called nodes, 
where each node has itself a value, and points to the 
next node in the linked list. The principal advantage 
of a linked list over an array, is that values can always 
be efficiently inserted and removed without relocating 
the rest of the list. Certain other operations, such as 
random access to a certain element, are however 
slower on lists than on arrays. Most assembly languages and some low-level languages, such as BCPL (Basic Combined Programming Language), 
lack built-in support for data structures. On the other 
hand, many high-level programming languages and 
some higher-level assembly languages, such as 
MASM, have special syntax or other built-in support 
for certain data structures, such as records and arrays.

\section{SEQUENTIAL SEARCH}
When data items are stored in a collection such as a 
list, we say that they have a linear or sequential 
relationship. Each data item is stored in a position 
relative to the others. In Python lists, these relative 
positions are the index values of the individual items. 
Since these index values are ordered, it is possible for 
us to visit them in sequence. This process gives rise to 
our first searching technique, the sequential search. 
Starting at the first item in the list, we simply move 
from item to item, following the underlying sequential 
ordering until we either find what we are looking for 
or run out of items. If we run out of items, we have 
discovered that the item we were searching for was 
not present.



\begin{table}
\section*{ALGORITHM COMPLEXITY}
\begin{center}
    
    \centering
    \begin{tabular}{ | m{10em} | m{10em}| m{10em} | } 
    \hline
         Algorithm & Best case & Expected \\
         \hline
         Selection sort & O(N2) & O(N2)\\
         \hline
         Merge sort & O(NlogN) & O(NlogN)\\
         \hline
         Linear search & O(1) & O(1)\\
         \hline
         Binary search & O(1) & O(1)\\
         \hline
         
         
    \end{tabular}
  
    \label{tab:my_label}
    \end{center}
\end{table}

\section*{DEPTH OF NODE}
The depth of node is the length of the path from the 
root to the node. A rooted tree with only one node has a depth of zero.

\section*{MATHEMATICAL EXPRESSIONS}
This is a simple math expression \(\sqrt{x^2+1}\) inside text. \\
And this is also the same: 
\begin{math}
\sqrt{x^2+1}
\end{math} \\
Ths is an inline matrix here:
$\begin{pmatrix}
  a & b\\ 
  c & d
\end{pmatrix}$ 

\section*{Insertion of Figure}
This is a figure: \\
\setlength{\unitlength}{0.8cm}
\begin{picture}(12,4)
\thicklines
\put(8,3.3){{\footnotesize $3$-simplex}}
\put(9,3){\circle*{0.1}}
\put(8.3,2.9){$a_2$}
\put(8,1){\circle*{0.1}}
\put(7.7,0.5){$a_0$}
\put(10,1){\circle*{0.1}}
\put(9.7,0.5){$a_1$}
\put(11,1.66){\circle*{0.1}}
\put(11.1,1.5){$a_3$}
\put(9,3){\line(3,-2){2}}
\put(10,1){\line(3,2){1}}
\put(8,1){\line(1,0){2}}
\put(8,1){\line(1,2){1}}
\put(10,1){\line(-1,2){1}}
\end{picture}
\\ This is another figure: \\
\setlength{\unitlength}{1cm}
\thicklines
\begin{picture}(10,6)
\put(2,2.2){\line(1,0){6}}
\put(2,2.2){\circle{2}}
\put(6,2.2){\oval(4,2)[r]}
\end{picture}

\section*{Reference}
\subsection{Book of Data structures through C G. S Baluja.}
\subsection{Pieren Garry Department of computer science New York University.}
\subsection{Paul Xavier department of algorithms in Amsterdam}
\subsection{Surendrakumar Ahuja IItdelhi department of computer science delhi}
\subsection{Nick jones department of data mining Australia}
\subsection{Wikipedia sequential search.}
\end{document}
